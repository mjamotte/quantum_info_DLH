\documentclass[aspectratio=169,12pt]{beamer}
\usetheme{Madrid}
\usecolortheme{dolphin}
\usefonttheme{professionalfonts}
\setbeamertemplate{navigation symbols}{}
\setbeamertemplate{footline}{}

\usepackage[T1]{fontenc}
\usepackage[utf8]{inputenc}
\usepackage{graphicx}
\usepackage{amsmath,amssymb}
\usepackage{hyperref}
\usepackage{siunitx}
\usepackage{physics}

\title[Chapter 2]{Chapitre 2: Notions essentielles de mécanique quantique appliquées à l'informatique}
\author{JAMOTTE Maxime, SCHOONEN Cédric}
\institute{Digital Learning Hub}
\date{}

\begin{document}

\begin{frame}
  \titlepage
\end{frame}

\begin{frame}{Table des matières}
  \tableofcontents
\end{frame}

\section{Vecteurs, matrices et produit matriciel}

\begin{frame}{Vecteurs}
  {%
    \setbeamercolor{itemize/enumerate body}{fg=gray!60}
    \setbeamercolor{itemize item}{fg=gray!60}
    \setbeamercolor{alerted text}{fg=black}
    \begin{columns}[T]
      \begin{column}{0.55\textwidth}
        \begin{itemize}[<+-| alert@+>]
          \setlength{\itemsep}{0.6em}
          \item Définition d'un vecteur
          \item Addition de vecteurs
          \item Multiplication d'un vecteur par une constante
        \end{itemize}
        \bigskip
        \only<1>{\[
          \vec r = \begin{pmatrix} a \\ b \end{pmatrix}
        \]}
        \only<2>{\[
          \vec r + \vec p = 
          \begin{pmatrix} a \\ b \end{pmatrix} +
          \begin{pmatrix} c \\ d \end{pmatrix} =
          \begin{pmatrix} a+c \\ b+d \end{pmatrix}
        \]}
        \only<3>{\[
          \lambda \begin{pmatrix} a \\ b \end{pmatrix} =
          \begin{pmatrix} \lambda a \\ \lambda b \end{pmatrix}
        \]}
      \end{column}
      \begin{column}{0.4\textwidth}
        \centering
        \only<1>{\includegraphics[width=\linewidth]{figures/vecteur.png}}
        \only<2>{\includegraphics[width=\linewidth]{figures/add_vecteurs.png}}
        \only<3->{\includegraphics[width=\linewidth]{figures/multiplication_nombre.png}}
      \end{column}
    \end{columns}
  }
\end{frame}

\begin{frame}{Matrices}
  {%
    \setbeamercolor{itemize/enumerate body}{fg=gray!60}
    \setbeamercolor{itemize item}{fg=gray!60}
    \setbeamercolor{alerted text}{fg=black}
    \begin{columns}[T]
      \begin{column}{0.55\textwidth}
        \begin{itemize}[<+-| alert@+>]
          \setlength{\itemsep}{0.6em}
          \item Addition de matrices
          \item Multiplication d'une matrice par un scalaire
          \item Multiplication matrice-vecteur
        \end{itemize}
        \bigskip
        \only<1>{\[M_1^{}+M_2^{} = 
          \begin{pmatrix} a & b\\ c & d \end{pmatrix} +
          \begin{pmatrix} e & f\\ g & h \end{pmatrix} =
          \begin{pmatrix} a+e & b+f\\ c+g & d+h \end{pmatrix}
        \]}
        \only<2>{\[ \lambda M = 
          \lambda \begin{pmatrix} a & b\\ c & d \end{pmatrix} =
          \begin{pmatrix} \lambda a & \lambda b\\ \lambda c & \lambda d \end{pmatrix}
        \]}
        \only<3>{\[M \vec{r} = 
          \begin{pmatrix} a & b \\ c & d \end{pmatrix}
          \begin{pmatrix} u \\ v \end{pmatrix} =
          \begin{pmatrix} au + bv \\ cu + dv \end{pmatrix}
        \]}
      \end{column}
      \begin{column}{0.4\textwidth}
        \centering
        \only<1>{\fbox{\Large $M_1^{}+M_2^{}$}}
        \only<2>{\fbox{\Large $\lambda M$}}
        \only<3->{\includegraphics[width=\linewidth]{figures/multiplication_Mv.png}}
      \end{column}
    \end{columns}
  }
\end{frame}

\section{Qubit et états superposés}

\begin{frame}{Pourquoi des qubits ?}
  \begin{itemize}[<+->]
    \setlength{\itemsep}{0.6em}
    \item Certains systèmes physiques n'ont que deux états possibles : ils servent de bits quantiques.
    \item Postulat : les états quantiques forment un espace vectoriel (espace de Hilbert).
    \item Un qubit peut être dans $\ket 0$, $\ket 1$ ou toute superposition des deux.
  \end{itemize}
\end{frame}

\begin{frame}{Superposition d'états}
  \begin{itemize}[<+->]
    \setlength{\itemsep}{0.6em}
    \item État général : $\alpha \ket 0 + \beta \ket 1$ avec $|\alpha|^2 + |\beta|^2 = 1$.
    \item Probabilités de mesure : $P(0)=|\alpha|^2$, $P(1)=|\beta|^2$.
    \item Mesure aléatoire : parfois $0$, parfois $1$, selon ces probabilités.
  \end{itemize}
\end{frame}

\begin{frame}{Notations pratiques}
  \begin{itemize}[<+->]
    \setlength{\itemsep}{0.6em}
    \item On note aussi $\ket 0 = \begin{pmatrix}1 \\ 0\end{pmatrix}$ et $\ket 1 = \begin{pmatrix}0 \\ 1\end{pmatrix}$.
    \item Une superposition $\alpha \ket 0 + \beta \ket 1$ s'écrit $\begin{pmatrix} \alpha \\ \beta \end{pmatrix}$.
    \item Rappel : amplitudes complexes, probabilités données par le module au carré.
  \end{itemize}
\end{frame}

\section{Portes logiques classique et quantiques}

\begin{frame}{References}
  \footnotesize
  \begin{thebibliography}{9}
    \bibitem{ref:placeholder} Author, \textit{Title}, Venue (Year).
  \end{thebibliography}
\end{frame}

\end{document}
