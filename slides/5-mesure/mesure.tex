\documentclass[aspectratio=169,12pt]{beamer}
\usetheme{Madrid}
\usecolortheme{dolphin}
\usefonttheme{professionalfonts}
\setbeamertemplate{navigation symbols}{}
\setbeamertemplate{footline}{}

\usepackage[T1]{fontenc}
\usepackage[utf8]{inputenc}
\usepackage{graphicx}
\usepackage{amsmath,amssymb}
\usepackage{hyperref}
\usepackage{siunitx}
\usepackage{physics}
\usepackage{braket}

\title[Chapter 5]{Chapitre 5: Vecteurs propres, valeurs propres, mesure et probabilité}
\author{JAMOTTE Maxime, SCHOONEN Cédric}
\institute{Digital Learning Hub}
\date{}

\begin{document}

\begin{frame}
  \titlepage
\end{frame}

\begin{frame}{Objectifs du chapitre 5}
    \begin{itemize}
        \item Vecteurs propres et valeurs propres\\~\\
        \item Comprendre les liens entre "valeur propre/résultat de mesure" et "vecteur propre/probabilité de mesure"\\~\\
        \item Mesurer avec QISKIT
    \end{itemize}
\end{frame}

\begin{frame}{Valeurs propres et vecteurs propres}
  \begin{itemize}[<+->]
    \setlength{\itemsep}{0.6em}
    \item Cas particuliers d'un produit matrice-vecteur:   
    $$ \begin{pmatrix} 2 & 1 \\ 2 & 3 \end{pmatrix} \begin{pmatrix} -1 \\ 1 \end{pmatrix} = \begin{pmatrix} -1 \\ 1 \end{pmatrix},$$
    et 
    $$ \begin{pmatrix} 2 & 1 \\ 2 & 3 \end{pmatrix} \begin{pmatrix} 1 \\ 2 \end{pmatrix} = \begin{pmatrix} 4 \\ 8 \end{pmatrix} = 4 \begin{pmatrix} 1 \\ 2 \end{pmatrix},$$~\\
    \item Multiplier ce vecteur par cette matrice redonne le même vecteur multiplié par un nombre.
    \item On nomme ce vecteur "\textbf{vecteur propre}" et ce nombre "\textbf{valeur propre}".
  \end{itemize}
\end{frame}

\begin{frame}{Lien entre valeurs propres, vecteurs propres, mesure et probabilité}
  \begin{itemize}[<+->]
    \setlength{\itemsep}{0.6em}
    \item Les grandeurs mesurables, comme la position, l'énergie ou le spin, sont repésentées par des matrices,
     appelées \textit{observables}, et \textbf{mesurer} une quantité physique revient en fait à mesurer
      l'une de ses \textbf{valeurs propres}.
    \item La nature physique de la valeur propre (énergie, spin,...) dépend de la platforme (voir plus loin).
    \item La \textbf{probabilité} de mesurer une valeur propre est directement reliée
     au \textbf{produit scalaire} entre l'état du système et le vecteur propre associé à la valeur propre mesurée.
  \end{itemize}
\end{frame}

\begin{frame}{Produit scalaire}
  \begin{columns}[T]
    \begin{column}{0.6\textwidth}
      \begin{itemize}
        \setlength{\itemsep}{0.6em}
        \item La probabilité d'obtenir un résultat de mesure vient (en gros) d'une comparaison de deux états quantiques.
        \item Cette comparaison se traduit par un produit scalaire entre leurs vecteurs associés.
        $$\vec r \cdot \vec p = \begin{pmatrix} a & b \end{pmatrix} \begin{pmatrix} c \\ d \end{pmatrix} = ac+bd$$
        \item Lien avec l'angle: $\vec r \cdot \vec p = \norm{\vec r}\norm{\vec p}\cos(\theta)$.
      \end{itemize}
    \end{column}
    \begin{column}{0.35\textwidth}
      \centering
      \includegraphics[width=\linewidth]{figures/proj.png}
    \end{column}
  \end{columns}
\end{frame}

\begin{frame}{Mesurer en mécanique quantique}
  \begin{itemize}
    \item Produit scalaire entre deux états $\ket{\phi}$ et $\ket \psi$: $\braket{\phi|\psi}$.
    \item Conventions: $\braket{0|0} = 1$, $\braket{1|1} = 1$ et $\braket{0|1} = 0$ (orthogonaux).
    \item Si $\ket \psi = a \ket 0 + b \ket 1$ et $\ket \phi = c \ket 0 + d \ket 1$ alors:
    \begin{equation*}
      \begin{split}
        \braket{\phi|\psi}
        &= \begin{pmatrix}
          a\  b
        \end{pmatrix}
        \begin{pmatrix}
          c\\d
        \end{pmatrix} = ac \braket{0|0} + ad \braket{1|0} + bc \braket{0|1} + bd \braket{1|1} = ac +bd\\
        &= \vec r_\phi^{} \cdot \vec p_\psi^{} = \|\vec r_\phi^{}\|\ \|\vec p_\psi^{}\| \cos(\theta)
      \end{split}
    \end{equation*}
    \item \textbf{Postulat}: mesurer = projeter le vecteur du qubit sur le vecteur propre associé à la
     valeur propre de l'observable mesurée.
     \item On projette avec un projecteur $\Pi_{\phi}^{} = \ket \phi \bra \phi$ sur l'état $\phi$: $\ket \phi \braket{\phi|\psi}$
  \end{itemize}  
\end{frame}


\begin{frame}{Mesurer avec QISKIT}
  \begin{columns}[T,onlytextwidth]
    \column{0.5\textwidth}
    \begin{itemize}
      \item Ajouter un bit classique pour stocker la mesure: \texttt{qc(1,1)}.
      \item Opération de mesure et stockage dans le bit classique: \texttt{qc.measure(0, 0)}.
      \item Probabilités obtenues par le module carré des amplitudes: si $\ket \psi = \ket 0$ ou $\ket 1$, $$ P(\ket 0)=
      |\braket{0|H|\psi}|^2 = 1/2,$$  $$P(\ket 1)=|\braket{1|H|\psi}|^2 = 1/2$$ ($\braket{\cdot|\cdot} = $ produit scalaire entre l'état mesuré et l'état du circuit).
    \end{itemize}
    \column{0.45\textwidth}
    \centering
    \vspace*{-3em}
    \includegraphics[width=0.9\linewidth]{figures/measure_circuit.png}
    \\[0.05em]
    \vspace*{-1.5em}
    \[
      \begin{aligned}
        \braket{0|H|\psi} &= \begin{pmatrix} 1\\0 \end{pmatrix} \cdot \frac{1}{\sqrt 2}\begin{pmatrix} 1\\\pm 1 \end{pmatrix} = \frac{1}{\sqrt 2},\\
        \braket{1|H|\psi} &= \begin{pmatrix} 0\\1 \end{pmatrix} \cdot \frac{1}{\sqrt 2}\begin{pmatrix} 1\\\pm 1 \end{pmatrix} = \pm \frac{1}{\sqrt 2}.
      \end{aligned}
    \]
  \end{columns}
\end{frame}

\begin{frame}{Mesurer avec QISKIT }
  \begin{columns}[T,onlytextwidth]
    \column{0.5\textwidth}
    \begin{itemize}
     \item Ajouter un bit classique et mesurer: \texttt{qc.measure(0, 0)}.
      \item Le résultat de mesure doit être stocké dans le registre classique pour être lisible.
      \item Probabilités obtenues par le module carré des amplitudes: si $\ket \psi = \ket 0$ ou $\ket 1$, $$ P(\ket 0)=
      |\braket{0|H|\psi}|^2 = 1/2,$$  $$P(\ket 1)=|\braket{1|H|\psi}|^2 = 1/2$$
    \end{itemize}
    \column{0.45\textwidth}
    \centering
    \vspace*{-3em}
    \includegraphics[width=0.9\linewidth]{figures/measure_circuit.png}
    \\[0.05em]
    \vspace*{-1.5em}
    \[
      \begin{aligned}
        \braket{0|H|\psi} &= \begin{pmatrix} 1\\0 \end{pmatrix} \cdot \frac{1}{\sqrt 2}\begin{pmatrix} 1\\\pm 1 \end{pmatrix} = \frac{1}{\sqrt 2},\\
        \braket{1|H|\psi} &= \begin{pmatrix} 0\\1 \end{pmatrix} \cdot \frac{1}{\sqrt 2}\begin{pmatrix} 1\\\pm 1 \end{pmatrix} = \pm \frac{1}{\sqrt 2}.
      \end{aligned}
    \]
  \end{columns}
  \vspace*{-.5em}
  ~\\~\\La mesure \textbf{projette} l'état quantique sur $\ket{0}$ ou $\ket{1}$
   selon les probabilités $P(\ket 0)$ et $P(\ket 1)$. \textbf{Le système est définitivement dans un des états propres}.
\end{frame}

\begin{frame}{Simulation avec Aer: imitation d'un ordinateur quantique}
  \begin{itemize}
    \item Backend: \texttt{simulateur = Aer.get\_backend('aer\_simulator')}.
    \item Exécuter avec un nombre de tirs \texttt{shots = n}, récupérer les comptages.
    \item Visualiser les résultats avec \texttt{plot\_histogram}.
  \end{itemize}
  \vspace{0.5em}
  \begin{columns}[T,onlytextwidth]
    \column{0.5\textwidth}
    \centering
    \includegraphics[width=0.8\linewidth]{figures/histogram_1_shot.png}\\[-0.3em]
    \footnotesize 1 tir
    \column{0.5\textwidth}
    \centering
    \includegraphics[width=0.8\linewidth]{figures/histogram_1000_shots.png}\\[-0.3em]
    \footnotesize 1000 tirs
  \end{columns}
\end{frame}

\begin{frame}{Fin du chapitre 5}
    \begin{itemize}
        \item Vecteurs propres et valeurs propres\\~\\
        \item Comprendre les liens entre "valeur propre/résultat de mesure" et "vecteur propre/probabilité de mesure"\\~\\
        \item Mesurer avec QISKIT
    \end{itemize}
\end{frame}


\end{document}
