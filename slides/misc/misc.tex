\documentclass[aspectratio=169,12pt]{beamer}
\usetheme{Madrid}
\usecolortheme{dolphin}
\usefonttheme{professionalfonts}
\setbeamertemplate{navigation symbols}{}
\setbeamertemplate{footline}{}

\usepackage[T1]{fontenc}
\usepackage[utf8]{inputenc}
\usepackage{graphicx}
\usepackage{amsmath,amssymb}
\usepackage{hyperref}
\usepackage{siunitx}
\usepackage{physics}

\begin{document}
	
	% ============================ NOISE CHAPTER ============================
	% Noise is a major obstacle to quantum computation
	% Types: gate errors and decoherence (what is the difference?)
	% Brief discussion about decoherence ??
	% Currently in the "NISQ" era: cannot yet do Grover and Shor at realistic scales
	% There are algorithms more tolerant to a noisy environment: VQE and QAOA 
	% --> why are they more tolerant to noise? (brief explanation)
	% Currently no error correction (or little, see Microsoft April 2024)
	% Talk briefly about error correction strategies
	% Roadmaps of quantum computing companies
	% Even if you don't share their optimistim, it shows the importance of migrating to quantum-safe cryptographic protocols
	% =======================================================================
	
	
	% ============================= VQA CHAPTER =============================
	
	
	% Méthodes hybrides consistant à minimizer avec des algorithmes classiques une fonction coût V(q) = <q|H|q> évaluée sur un ordinateur quantique.
	% ... méthode variationelle en physique (<q|H|q> >= ground state energy)
	% ... make a diagram with boxes for initialization of parameters q, eval of <q|H|q>, result V(q), arrows in between + one retro-acting from V(q) to the initialization q
	% Expliquer comment cela est lié à l'estimation de phase
	% ((Passer en revue les nombreuses variantes: VQE, QAOA, QML)) 
	%   --> QAOA IS NOT A VARIATIONAL ALGO! AND QML IS A FIELD!!
	%   --> why presented as such in qss slides
	% --------------------------------------------------------
	% VQE (Variational Quantum Eigensolver):
	% - Main applications in chemistry and condensed matter physics
	% - Reférence 2013 sur slides QGSS-2025
	% QAOA (Quantum Approximate Optimization Algorithms):
	% - Combinatorial optimization with general applications (e.g. logistics with the travelling salesman problem)
	% - Un peu compliqué à expliquer (mapping problem to a hamiltonian H1) (adiabatic from H0 with known ground state to H1) (trotterization)
	% - Reférence 2014 sur slides QGSS-2025
	% QML (Quantum Machine Learning):
	% - Classifiers with Support Vector Machines (can be formulated as an optimization problem)
	% - Neural network training
	% - Benefits from HHL algo --> how?
	% --------------------------------------------------------
	% Expliquer ce que la partie quantique accélère exactement
	% Expliquer les problèmes associés à l'étape (classique) de minimisation: barren plateaus (expliquer argument de base -> papier 2018)
	% Astuces pour éviter ces problèmes: initialisations bien choisies avec un nombre réduit d'opérateurs (ADAPT-VQE arxiv:1812.11173) (Aussi arxiv:2401.08044 arxiv:2308.04481)
	% Question ouverte: est-ce qu'éviter le problème des barren plateaus revient à rendre l'algorithme simulable classiquement? (arxiv:2312.09121)

\begin{frame}{Classique VS Quantique}
    \begin{itemize}
        \item Calcul représenté comme une opération sur une séquence de bits $(b_n \cdots b_2 b_1)$
        \item Ordinateur classique opère sur une séquence à la fois: %$(b_n,\cdots,b_2,b_1) \longrightarrow f(b_n,\cdots,b_2,b_1)$
        \begin{align*}
            (101) \longrightarrow f(101)
        \end{align*}
        \item Ordinateur quantique peut encoder toutes les séquences possibles dans un seul état et opérer sur toutes les séquences en une seule opération:
        \begin{align*}
            \begin{Bmatrix}
                (000) \\ (001) \\ (010) \\ \cdots \\ (111)
            \end{Bmatrix}
            \quad \longrightarrow \quad
            \begin{Bmatrix}
                f(000) \\ f(001) \\ f(010) \\ \cdots \\ f(111)
            \end{Bmatrix}
        \end{align*}
        \item Problème: on ne peut récupérer qu'un seul des résultats...
    \end{itemize}
\end{frame}

% Version more suitable for an exercise
\begin{frame}{No-Cloning$^*$}
    \begin{itemize}
        \item Supposons qu'une opération $U$ existe pour copier des états génériques $|\phi \rangle$ et $|\psi \rangle$:
        \begin{align*}
            U(|\phi \rangle _{A}|e\rangle _{B}) &= e^{i\alpha (\phi ,e)}|\phi \rangle _{A}|\phi \rangle _{B} \\
            U(|\psi \rangle _{A}|e\rangle _{B}) &= e^{i\alpha (\psi ,e)}|\psi \rangle _{A}|\psi \rangle _{B}
        \end{align*}
        \item Le calcul du produit scalaire des états clonés mène à deux résultats contradictoires:
        \begin{align*}
            | \langle \phi|_{A}\langle e|_{B} \, U^\dagger U \, |\psi \rangle _{A}|e\rangle _{B} | &= | \langle \phi |\psi \rangle |^{2} \\
            | \langle \phi|_{A}\langle e|_{B} \, U^\dagger U \, |\psi \rangle _{A}|e\rangle _{B} | &= | \langle \phi |\psi \rangle |
        \end{align*}
        \item Cela implique que $|\langle \phi |\psi \rangle |=1$ ou $|\langle \phi |\psi \rangle |=0$, en contradiction avec l'hypothèse de généralité des états $|\phi \rangle$ et $|\psi \rangle$.
    \end{itemize}
\end{frame}

% Detailed version
\begin{frame}{No-Cloning}
    \begin{itemize}
        \item Supposons qu'une opération $U$ existe pour copier des états génériques $|\phi \rangle$ et $|\psi \rangle$:
        \begin{align*}
            U(|\phi \rangle _{A}|e\rangle _{B}) &= e^{i\alpha (\phi ,e)}|\phi \rangle _{A}|\phi \rangle _{B} \\
            U(|\psi \rangle _{A}|e\rangle _{B}) &= e^{i\alpha (\psi ,e)}|\psi \rangle _{A}|\psi \rangle _{B}
        \end{align*}
        \item En prenant le produit scalaire des deux états clonés, on obtient:
        \begin{align*}
            \langle \phi|_{A}\langle e|_{B} \, U^\dagger U \, |\psi \rangle _{A}|e\rangle _{B} &= e^{i[\alpha (\psi ,e)-\alpha (\phi ,e)]}\langle \phi |\psi \rangle ^{2} \\
            \langle \phi|_{A}\langle e|_{B} \, U^\dagger U \, |\psi \rangle _{A}|e\rangle _{B} &= \langle \phi |\psi \rangle \langle e |e \rangle
        \end{align*}
        \item Mais alors $|\langle \phi |\psi \rangle| = |\langle \phi |\psi \rangle|^{2}$, ce qui implique que $|\langle \phi |\psi \rangle |=1$ ou $|\langle \phi |\psi \rangle |=0$, en contradiction avec l'hypothèse de généralité des états $|\phi \rangle$ et $|\psi \rangle$.
    \end{itemize}
\end{frame}

\begin{frame}{Transpilation : Six Étapes}
  \begin{columns}[T]
    \begin{column}{0.58\textwidth}
      \small
      \vspace{-0.7em}
      \begin{enumerate}
        \setlength{\itemsep}{0.3em}
        \item \textbf{Init}: Réduction des opérations multi-qubits en portes à 1 et 2 qubits (Optionnel)
        \item \textbf{Layout}: Assignation des qubits physiques
        \item \textbf{Routing}: Insertion de portes SWAP pour déplacer les états quantiques entre qubit (NP-difficile, algorithme aléatoire)
        \item \textbf{Translation}: Conversion de toutes les portes vers le jeu d'instructions natif (ISA)
        \item \textbf{Optimization}: Remplacement ou élimination de portes logiques où c'est possible
        \item \textbf{Scheduling}: Insertion de délais (Optionnel)
      \end{enumerate}
    \end{column}
    \begin{column}{0.4\textwidth}
      \centering
      \includegraphics[width=\textwidth]{../../figures/qubits_layout.jpg}
    \end{column}
  \end{columns}
\end{frame}

\begin{frame}{Transpilation: Niveaux d'Optimisation}
  \small
  \begin{description}
    \setlength{\itemsep}{1.5em}
    \item[\textbf{Niveau 1}:] Combination matricielle des portes à un qubit,\\ élimination des portes CNOT redondantes
    
    \item[\textbf{Niveau 2}:] Analyse de commutation pour éliminer davantage de redondances
    
    \item[\textbf{Niveau 3}:] Optimisation des sous-blocs à deux qubits identifiés dans le circuit\\ (reconstruit une matrice unitaire équivalente)
  \end{description}
\end{frame}

\end{document}
