\documentclass[aspectratio=169,12pt]{beamer}
\usetheme{Madrid}
\usecolortheme{dolphin}
\usefonttheme{professionalfonts}
\setbeamertemplate{navigation symbols}{}
\setbeamertemplate{footline}{}

\usepackage[T1]{fontenc}
\usepackage[utf8]{inputenc}
\usepackage{graphicx}
\usepackage{amsmath,amssymb}
\usepackage{hyperref}
\usepackage{siunitx}
\usepackage{physics}

% \title[Chapter 2]{Chapitre 2: Notions essentielles de mécanique quantique appliquées à l'informatique}
% \author{JAMOTTE Maxime, SCHOONEN Cédric}
% \institute{Digital Learning Hub}
% \date{}

\begin{document}

% Version more suitable for an exercise
\begin{frame}{No-Cloning$^*$}
    \begin{itemize}
        \item Supposons qu'une opération $U$ existe pour copier des états génériques $|\phi \rangle$ et $|\psi \rangle$:
        \begin{align*}
            U(|\phi \rangle _{A}|e\rangle _{B}) &= e^{i\alpha (\phi ,e)}|\phi \rangle _{A}|\phi \rangle _{B} \\
            U(|\psi \rangle _{A}|e\rangle _{B}) &= e^{i\alpha (\psi ,e)}|\psi \rangle _{A}|\psi \rangle _{B}
        \end{align*}
        \item Le calcul du produit scalaire des états clonés mène à deux résultats contradictoires:
        \begin{align*}
            | \langle \phi|_{A}\langle e|_{B} \, U^\dagger U \, |\psi \rangle _{A}|e\rangle _{B} | &= | \langle \phi |\psi \rangle |^{2} \\
            | \langle \phi|_{A}\langle e|_{B} \, U^\dagger U \, |\psi \rangle _{A}|e\rangle _{B} | &= | \langle \phi |\psi \rangle |
        \end{align*}
        \item Cela implique que $|\langle \phi |\psi \rangle |=1$ ou $|\langle \phi |\psi \rangle |=0$, en contradiction avec l'hypothèse de généralité des états $|\phi \rangle$ et $|\psi \rangle$.
    \end{itemize}
\end{frame}

% Detailed version
\begin{frame}{No-Cloning}
    \begin{itemize}
        \item Supposons qu'une opération $U$ existe pour copier des états génériques $|\phi \rangle$ et $|\psi \rangle$:
        \begin{align*}
            U(|\phi \rangle _{A}|e\rangle _{B}) &= e^{i\alpha (\phi ,e)}|\phi \rangle _{A}|\phi \rangle _{B} \\
            U(|\psi \rangle _{A}|e\rangle _{B}) &= e^{i\alpha (\psi ,e)}|\psi \rangle _{A}|\psi \rangle _{B}
        \end{align*}
        \item En prenant le produit scalaire des deux états clonés, on obtient:
        \begin{align*}
            \langle \phi|_{A}\langle e|_{B} \, U^\dagger U \, |\psi \rangle _{A}|e\rangle _{B} &= e^{i[\alpha (\psi ,e)-\alpha (\phi ,e)]}\langle \phi |\psi \rangle ^{2} \\
            \langle \phi|_{A}\langle e|_{B} \, U^\dagger U \, |\psi \rangle _{A}|e\rangle _{B} &= \langle \phi |\psi \rangle \langle e |e \rangle
        \end{align*}
        \item Mais alors $|\langle \phi |\psi \rangle| = |\langle \phi |\psi \rangle|^{2}$, ce qui implique que $|\langle \phi |\psi \rangle |=1$ ou $|\langle \phi |\psi \rangle |=0$, en contradiction avec l'hypothèse de généralité des états $|\phi \rangle$ et $|\psi \rangle$.
    \end{itemize}
\end{frame}

\end{document}
