\documentclass[aspectratio=169,12pt]{beamer}
\usetheme{Madrid}
\usecolortheme{dolphin}
\usefonttheme{professionalfonts}
\setbeamertemplate{navigation symbols}{}
\setbeamertemplate{footline}{}

\usepackage[T1]{fontenc}
\usepackage[utf8]{inputenc}
\usepackage{graphicx}
\usepackage{amsmath,amssymb}
\usepackage{hyperref}
\usepackage{siunitx}
\usepackage{physics}

\title[Inégalités de Bell]{Les inégalités de Bell}
\author{JAMOTTE Maxime, SCHOONEN Cédric}
\institute{Digital Learning Hub}
\date{}

\begin{document}

\begin{frame}
  \titlepage
\end{frame}

\begin{frame}{Table des matières}
  \tableofcontents
\end{frame}

\section{Paradoxe EPR}

\begin{frame}{Paradoxe EPR: Situation}
  \begin{columns}[T]
    \begin{column}{0.55\textwidth}
      États intriqué produit par une source et partagé entre Alice et Bob:
      \[
        \ket{\psi_{AB}} = \frac{\ket{00} + \ket{11}}{\sqrt{2}} \sim \frac{\ket{++} + \ket{--}}{\sqrt{2}}
      \]
    \end{column}
    \begin{column}{0.4\textwidth}
      \centering
      \includegraphics[width=\linewidth]{../../figures/epr.png}
    \end{column}
  \end{columns}
\end{frame}

\begin{frame}{Mesures de Bob seul}
  Si Bob est seul à effectuer des mesures, il a toujours $50\%$ de chance d'obtenir un des deux états possibles ($\ket 0$ ou $\ket 1$ en base $Z$, et $\ket +$ ou $\ket -$ en base $X$).
  
  \bigskip
  Les valeurs moyennes mesurées sont donc nulles:
  \[
    \langle Z_B \rangle = \left( \frac{\bra{00} + \bra{11}}{\sqrt{2}} \right) \left( \frac{\ket{00} + (-1) \ket{11}}{\sqrt{2}} \right) = 0
  \]
  \[
    \langle X_B \rangle = \left( \frac{\bra{00} + \bra{11}}{\sqrt{2}} \right) \left( \frac{\ket{01} + \ket{10}}{\sqrt{2}} \right) = 0
  \]
\end{frame}

\begin{frame}{Corrélations avec Alice}
  En revanche, si Alice effectue des mesures en base $Z$, Bob constatera (une fois qu'Alice lui aura partagé ses résultats), que ses résultats sont totalement corrélés avec ceux d'Alice.
  
  \bigskip
  Ses mesures effectuées dans la base $Z$ sont identiques à celles d'Alice, comme si les mesures d'Alice avaient influencé les siennes (ou vice-versa).
  \[
    \langle Z_AZ_B \rangle = \left( \frac{\bra{00} + \bra{11}}{\sqrt{2}} \right) \left( \frac{\ket{00} + (-1)^2 \ket{11}}{\sqrt{2}} \right) = 1
  \]
  \[
    \langle Z_AX_B \rangle = \left( \frac{\bra{00} + \bra{11}}{\sqrt{2}} \right) \left( \frac{\ket{01} +(-1) \ket{10}}{\sqrt{2}} \right) = 0
  \]
\end{frame}

\begin{frame}{Paradoxe EPR}
  \textbf{Question:} Comment les résultats de Bob peuvent-ils être corrélés avec ceux d'Alice, alors que Bob n'a aucune information sur les mesures effectuées par Alice?
  
  \bigskip
  Cela semble contredire le principe de \textit{localité} (aucune influence ne peut se propager plus vite que la lumière).

  \bigskip
  \textbf{Hypothèse d'EPR (1935):} La mécanique quantique est incomplète. Il existerait des variables cachées locales qui prédétermineraient les résultats des mesures.
\end{frame}

\section{Variables cachées}

\begin{frame}{Variables cachées}
  \textbf{Idée défendue:} Les probabilités $|\langle 0 | \psi \rangle|^2$ et $|\langle 1 | \psi \rangle|^2$ reflètent une incertitude sur la nature de l'état $\ket\psi$.
  
  \bigskip
  Il n'y aurait pas de superposition d'états, mais simplement des états dont nous ne connaissons pas la nature exacte (révélés lors de la mesure).
  
  \bigskip
  Une théorie plus fondamentale que la mécanique quantique décrirait exactement le résultat de la mesure d'une observable.
\end{frame}

\begin{frame}{Exemple de variable cachée}
  Prenons par exemple la superposition suivante:
  \[
    \ket\psi = \frac{\ket 0 + \ket 1}{\sqrt{2}}
  \]
  
  La probabilité de mesurer la valeur $+1$ pour l'observable $Z$ est:
  \[
    p(+1) = |\langle 0 | \psi \rangle|^2 = \frac 12
  \]
  
  L'hypothèse des variables cachées suppose que le résultat de la mesure de $Z$, $z = \pm 1$, serait pré-déterminé et encodé dans l'état quantique.
\end{frame}

\begin{frame}{Variable cachée (suite)}
  L'état quantique, $\ket{\psi^{(z)}}$, serait en réalité soit $\ket{\psi^{(+1)}} = \ket 0$ ou $\ket{\psi^{(-1)}} = \ket 1$.
  
  \bigskip
  La probabilité $1/2$ de mesurer $z=+1$ traduit simplement la probabilité que cette variable cachée soit égale à $+1$:
  \[
    p(+1) = |\langle 0 | \psi^{(z)} \rangle|^2 = 
    \begin{cases}
        1 \quad \text{ si } z=+1 \\
        0 \quad \text{ si } z=-1
    \end{cases}
    \quad
    = p(z=+1)
  \]
\end{frame}

\begin{frame}{Variables cachées et intrication}
  Les corrélations observées pour des états intriqués,
  \[
    \ket{\psi_{AB}} = \frac{\ket{00} + \ket{11}}{\sqrt{2}}
  \]
  \[
    \langle Z_AZ_B \rangle = \left( \frac{\bra{00} + \bra{11}}{\sqrt{2}} \right) \left( \frac{\ket{00} + (-1)^2 \ket{11}}{\sqrt{2}} \right) = 1
  \]
  
  seraient aussi expliquées par les valeurs pré-encodées de la variable cachée:
  \[
    \ket{\psi_{AB}} = 
    \begin{cases}
        \ket{\psi_A^{(+1)}}\otimes\ket{\psi_B^{(+1)}} \quad \text{ si } z=+1 \\
        \ket{\psi_A^{(-1)}}\otimes\ket{\psi_B^{(-1)}} \quad \text{ si } z=-1
    \end{cases}
  \]
\end{frame}

\section{Test statistique (CHSH)}

\begin{frame}{Test statistique (CHSH)}
  \begin{columns}[T]
    \begin{column}{0.55\textwidth}
      La source émet des états $\ket{\psi_A}$ et $\ket{\psi_B}$ vers Alice et Bob, respectivement.
      
      \bigskip
      Alice décide de mesurer une observable au choix parmi $A_0, A_1$, et Bob fait de même de son côté avec $B_0,B_1$.
      
      \bigskip
      Alice et Bob sont suffisamment éloignés pour que le résultat de l'un n'influence pas la mesure de l'autre.
    \end{column}
    \begin{column}{0.4\textwidth}
      \centering
      \includegraphics[width=\linewidth]{../../figures/chsh.png}
    \end{column}
  \end{columns}
\end{frame}

\begin{frame}{Analyse des corrélations}
  Les résultats des mesures ainsi que les observables choisies sont collectés et analysés à posteriori.
  
  \bigskip
  On calcule ensuite les corrélations $\langle A_iB_j \rangle$ et on évalue
  \[
    S = \langle A_0B_0 \rangle + \langle A_0B_1 \rangle + \langle A_1B_0 \rangle - \langle A_1B_1 \rangle
  \]
\end{frame}

\begin{frame}{Inégalité CHSH}
  S'il existe des variables cachées $a_0,a_1,b_0,b_1$ pré-déterminant les résultats de mesure de $A_0,A_1,B_0,B_1$, alors la combinaison moyenne $S$ sera évaluée à
  \begin{align*}
    S &= \sum_{a_0,a_1,b_0,b_1} p(a_0,a_1,b_0,b_1) \, \underbrace{(a_0b_0+ a_0b_1 + a_1b_0 - a_1b_1)}_{a_0(b_0+b_1) \, + \, a_1(b_0-b_1)}
  \end{align*}
  
  Comme $b_0,b_1 = \pm 1$ nous avons nécessairement un des deux facteurs $(b_0+b_1)$ ou $(b_0-b_1)$ qui s'annule, le deuxième prenant la valeur $\pm 2$.
\end{frame}

\begin{frame}{Inégalité CHSH (suite)}
  Ainsi,
  \[
    a_0(b_0+b_1) + a_1(b_0-b_1) = \pm 2
  \]
  
  et donc
  \[
    |S| \le 2
  \]
  
  \bigskip
  Cette équation s'appelle \textbf{l'inégalité CHSH} et est un exemple d'inégalité de Bell. Elle sert à tester l'hypothèse de présence de variables cachées.
\end{frame}

\section{Violation quantique}

\begin{frame}{La mécanique quantique ne respecte pas les inégalités de Bell}
  Dans le cas CHSH, considérons les observables suivantes pour Alice:
  \[
    A_0 = Z
    \qquad
    A_1 = X
  \]
  
  Et celles-ci pour Bob:
  \[
    B_0 = \frac{Z+X}{2}
    \qquad
    B_1 = \frac{Z-X}{2}
  \]
\end{frame}

\begin{frame}{Violation quantique (suite)}
  Si la source produit les états
  \[
    \ket{\psi_{AB}} = \frac{\ket{01} - \ket{10}}{\sqrt{2}}
  \]
  
  alors les corrélations seront
  \[
    \langle A_0B_0 \rangle = +\frac{1}{\sqrt{2}} \qquad
    \langle A_0B_1 \rangle = +\frac{1}{\sqrt{2}}
  \]
  \[
    \langle A_1B_0 \rangle = +\frac{1}{\sqrt{2}} \qquad
    \langle A_1B_1 \rangle = -\frac{1}{\sqrt{2}}
  \]
  
  Et la combinaison $S = 2\sqrt{2} > 2$
\end{frame}

\section{Expériences d'Alain Aspect}

\begin{frame}{Les expériences d'Alain Aspect (1982)}
  Implémentation concrète de la situation décrite dans le paradoxe EPR, avec de la lumière polarisée:
  
  \begin{center}
    \includegraphics[width=0.7\linewidth]{../../figures/aspect1.png}
  \end{center}
\end{frame}

\begin{frame}{Mise en place du test CHSH}
  Mise en place du test statistique CHSH:
  
  \begin{center}
    \includegraphics[width=0.7\linewidth]{../../figures/aspect2.png}
  \end{center}
\end{frame}

\begin{frame}{Interrupteur optique}
  Détail de l'interrupteur optique:
  
  \begin{center}
    \includegraphics[width=0.7\linewidth]{../../figures/aspect3.png}
  \end{center}
\end{frame}

\begin{frame}{Conclusions}
  \begin{itemize}
    \setlength{\itemsep}{0.8em}
    \item Les inégalités de Bell permettent de tester l'hypothèse des variables cachées locales
    \item La mécanique quantique prédit une violation de ces inégalités
    \item Les expériences d'Alain Aspect (1982) ont confirmé la violation
    \item Prix Nobel de physique 2022 pour Aspect, Clauser et Zeilinger
  \end{itemize}
\end{frame}

\begin{frame}{Vidéos sur le sujet}
  \begin{center}
    \includegraphics[width=\linewidth]{../../figures/scienceetonnante.png}
  \end{center}
\end{frame}

\end{document}
