\documentclass[aspectratio=169,12pt]{beamer}
\usetheme{Madrid}
\usecolortheme{dolphin}
\usefonttheme{professionalfonts}
\setbeamertemplate{navigation symbols}{}
\setbeamertemplate{footline}{}

\usepackage[T1]{fontenc}
\usepackage[utf8]{inputenc}
\usepackage{graphicx}
\usepackage{amsmath,amssymb}
\usepackage{hyperref}
\usepackage{siunitx}
\usepackage{physics}
\usepackage{braket}

\title[Chapter 4]{Chapitre ...: Plateformes implémentant des ordinateurs quantiques}
\author{JAMOTTE Maxime, SCHOONEN Cédric}
\institute{Digital Learning Hub}
\date{}

\begin{document}

\begin{frame}
  \titlepage
\end{frame}

\begin{frame}
    \begin{itemize}
        \item Plateforme classique\\~\\
        \item Plateformes quantiques
    \end{itemize}
\end{frame}

\begin{frame}{Plateforme classique : le silicium CMOS}
\begin{itemize}
    \item Les ordinateurs classiques utilisent des circuits \textbf{CMOS} fabriqués sur du \textbf{silicium}.
    \item Un bit est codé par un \textbf{niveau de tension} :
    \[
        0 = \text{tension basse}, \qquad 1 = \text{tension haute}.
    \]
    \item Le composant de base est le \textbf{transistor MOSFET} :
    \begin{itemize}
        \item canal ouvert = 1,
        \item canal fermé = 0.
    \end{itemize}
    \item Les portes logiques sont construites en reliant plusieurs transistors :
    \begin{itemize}
        \item NOT = 1 PMOS + 1 NMOS,
        \item AND/OR = transistors en série ou parallèle,
        \item NAND/NOR = briques fondamentales des processeurs.
    \end{itemize}
    \item Le tout est fabriqué par \textbf{photolithographie} sur du silicium dopé.
\end{itemize}
\end{frame}


%------------------ Slide 1 : Ions piégés ------------------%
\begin{frame}{Plateforme quantique : ions piégés}
\begin{itemize}
    \item Des \textbf{ions individuels} (ex.: Yb\(^+\), Ca\(^+\)) sont alignés dans un \textbf{piège électromagnétique}.
    \item Le qubit est encodé dans deux \textbf{niveaux internes} stables de l’atome :
    \[
        |0\rangle = \text{niveau hyperfin A}, \qquad |1\rangle = \text{niveau hyperfin B}.
    \]
    \item Les opérations quantiques sont réalisées par des \textbf{impulsions laser} :
    \begin{itemize}
        \item rotations à un qubit,
        \item ?portes à deux qubits via le \textbf{mode vibratoire collectif} (Molmer--Sørensen).
    \end{itemize}
    \item Plateformes : \textbf{Quantinuum}, \textbf{IonQ}, Innsbruck / AQT.
\end{itemize}
\end{frame}

%------------------ Slide 2 : Atomes neutres ------------------%
\begin{frame}{Plateforme quantique : atomes neutres}
\begin{itemize}
    \item Des atomes neutres (Rb, Cs) sont piégés un par un dans des \textbf{pinces optiques}.
    \item Le qubit correspond à deux \textbf{niveaux hyperfins} de l’atome :
    \[
        |0\rangle = \text{hyperfin A}, \qquad |1\rangle = \text{hyperfin B}.
    \]
    \item Manipulation par \textbf{lasers} :
    \begin{itemize}
        \item opérations à un qubit,
        \item ?interactions contrôlées via le \textbf{blocage de Rydberg}.
    \end{itemize}
    \item Les atomes peuvent être \textbf{déplacés librement} pour reconfigurer la géométrie du processeur.
    \item Plateformes : \textbf{Pasqal}, \textbf{QuEra}, ColdQuanta.
\end{itemize}
\end{frame}

%------------------ Slide 3 : Supraconducteurs ------------------%
\begin{frame}{Plateforme quantique : qubits supraconducteurs}
\begin{itemize}
    \item Des \textbf{circuits supraconducteurs} (LC + jonction Josephson) se comportent 
    comme des \textbf{atomes artificiels}.
    \item Les deux premiers niveaux d’énergie forment le qubit :
    \[
        |0\rangle = \text{état fondamental}, \qquad |1\rangle = \text{premier état excité}.
    \]
    \item Les opérations quantiques sont réalisées avec des \textbf{impulsions micro-ondes} :
    \begin{itemize}
        \item rotations à un qubit,
        \item portes contrôlées (\(CZ\), \(iSWAP\)).
    \end{itemize}
    \item Plateformes : \textbf{IBM Quantum}, \textbf{Google Quantum AI}, \textbf{Rigetti}.
\end{itemize}
\end{frame}

\end{document}



\end{document}