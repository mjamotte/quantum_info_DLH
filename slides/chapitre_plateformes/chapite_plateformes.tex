\documentclass[aspectratio=169,12pt]{beamer}
\usetheme{Madrid}
\usecolortheme{dolphin}
\usefonttheme{professionalfonts}
\setbeamertemplate{navigation symbols}{}
\setbeamertemplate{footline}{}

\usepackage[T1]{fontenc}
\usepackage[utf8]{inputenc}
\usepackage{graphicx}
\usepackage{amsmath,amssymb}
\usepackage{hyperref}
\usepackage{siunitx}
\usepackage{physics}
\usepackage{braket}
\usepackage{tikz}

\title[Chapter 4]{Chapitre 7: Plateformes implémentant des ordinateurs quantiques}
\author{JAMOTTE Maxime, SCHOONEN Cédric}
\institute{Digital Learning Hub}
\date{}

\begin{document}

\begin{frame}
  \titlepage
\end{frame}

\begin{frame}
    \begin{itemize}
        \item Plateforme classique\\~\\
        \item Plateformes quantiques
    \end{itemize}
\end{frame}

\begin{frame}{Plateforme classique : le silicium CMOS}
\begin{itemize}
    \item Les ordinateurs classiques utilisent des circuits \textbf{CMOS} fabriqués sur du \textbf{silicium}.
    \item Un bit est codé par un \textbf{niveau de tension} :
    \[
        0 = \text{tension basse}, \qquad 1 = \text{tension haute}.
    \]
    \item Le composant de base est le \textbf{transistor MOSFET} :
    \begin{itemize}
        \item canal ouvert = 1,
        \item canal fermé = 0.
    \end{itemize}
    \item Les portes logiques sont construites en reliant plusieurs transistors :
    \begin{itemize}
        \item NOT = 1 PMOS + 1 NMOS,
        \item AND/OR = transistors en série ou parallèle,
        \item NAND/NOR = briques fondamentales des processeurs.
    \end{itemize}
    \item Le tout est fabriqué par \textbf{photolithographie} sur du silicium dopé.
\end{itemize}
\end{frame}



\begin{frame}{Plateformes quantiques}
\begin{itemize}
    \item Qubits supraconducteurs\\~\\
    \item Atomes neutres\\~\\
    \item Ions piégés\\~\\
    \item Spins dans les semi-conducteurs\\~\\
    \item Photonique
\end{itemize}
\end{frame}

\begin{frame}{Qubits supraconducteurs (IBM, Google,...)}
\begin{columns}[T]
\begin{column}{0.6\textwidth}
\begin{itemize}
    \item Des \textbf{circuits supraconducteurs} (LC + jonction Josephson) se comportent 
    comme des \textbf{atomes artificiels}.\\~\\
    \item Les deux premiers niveaux d’énergie forment le qubit :
    \[
        |0\rangle = \text{état fondamental},
        |1\rangle = \text{premier état excité}.
    \]
    \item Les opérations quantiques sont réalisées avec des \textbf{impulsions laser micro-ondes}.
    ~\\~\\
    \item Fréquences différentes pour accéder aux autres niveaux d'énergie.
\end{itemize}
\end{column}
\begin{column}{0.4\textwidth}
\centering
\begin{tikzpicture}
    \node[inner sep=0] (img) {\includegraphics[width=0.8\linewidth]{figures/qubit_SC.png}};
    \node[anchor=south east,text=white,font=\scriptsize] at (img.south east) {Pierre Guichard};
\end{tikzpicture}
\end{column}
\end{columns}
\end{frame}

\begin{frame}{Ions piégés (Quantinuum, IonQ, Innsbruck / AQT)}
\begin{columns}[T]
\begin{column}{0.6\textwidth}
\begin{itemize}
    \item Des \textbf{ions individuels} (ex.: Yb\(^+\), Ca\(^+\)) sont alignés dans un \textbf{piège électromagnétique}.
    \item Le qubit est encodé dans deux \textbf{niveaux internes} stables de l’atome :
    \[
        |0\rangle = \text{niveau hyperfin A},\ |1\rangle = \text{niveau hyperfin B}.
    \]
    \item Opérations quantiques par des \textbf{impulsions laser}:
    \begin{itemize}
        \item rotations à un qubit,
        \item portes à deux qubits via \textbf{mode vibratoire collectif}.
    \end{itemize}
\end{itemize}
\end{column}
\begin{column}{0.4\textwidth}
\centering
\includegraphics[width=0.8\linewidth]{figures/ion_trap.jpg}
\end{column}
\end{columns}
\end{frame}

\begin{frame}{Atomes neutres (Pasqal, QuEra, ColdQuanta)}
\begin{columns}[T]
\begin{column}{0.6\textwidth}
\begin{itemize}
    \item Des atomes neutres (Rb, Cs) sont piégés un par un dans des \textbf{pinces optiques}.\\~\\
    \item Le qubit correspond à deux \textbf{niveaux hyperfins} de l’atome:
    \[
        |0\rangle = \text{hyperfin A}, \qquad |1\rangle = \text{hyperfin B}.
    \]
    \item Opérations quantiques par \textbf{impulsions laser}:
    \begin{itemize}
        \item opérations à un qubit,
        \item interactions contrôlées via le \textbf{blocage de Rydberg} (état hautement excité).\\~\\
    \end{itemize}
    \item Les atomes peuvent être \textbf{déplacés librement} pour reconfigurer la géométrie du processeur.
\end{itemize}
\end{column}
\begin{column}{0.4\textwidth}
\centering
\includegraphics[width=0.8\linewidth]{figures/atomes_neutres.png}
\end{column}
\end{columns}
\end{frame}

\begin{frame}{Fin du chapitre 7}
    \centering
 Passons à...
\end{frame}


\end{document}
