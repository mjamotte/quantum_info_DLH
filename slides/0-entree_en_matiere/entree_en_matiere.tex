\documentclass[aspectratio=169,12pt]{beamer}
\usetheme{Madrid}
\usecolortheme{dolphin}
\usefonttheme{professionalfonts}
\setbeamertemplate{navigation symbols}{}
\setbeamertemplate{footline}{}

\usepackage[T1]{fontenc}
\usepackage[utf8]{inputenc}
\usepackage{graphicx}
\usepackage{amsmath,amssymb}
\usepackage{hyperref}

\title[Entrée en matière]{Entrée en matière}
\author{JAMOTTE Maxime, SCHOONEN Cédric}
\institute{Digital Learning Hub}
\date{}

\begin{document}

\begin{frame}
  \titlepage
\end{frame}

\begin{frame}{Brisons la glace}
  \begin{columns}[T,onlytextwidth]
    \column{0.55\textwidth}
    \begin{itemize}
      \item Avez-vous bien signé la feuille de présence ?\\~\\
      \item Présentons-nous !\\~\\
      \item Vos motivations et attentes?
    \end{itemize}
    \column{0.45\textwidth}
    \centering
    \includegraphics[width=0.8\linewidth]{icebreaker.jpg}
  \end{columns}
\end{frame}

\begin{frame}{Supports du cours}
  \begin{itemize}
    \item Slides\\~\\
    \item Jupyter notebooks (Python - Qiskit)\\~\\
    \item Exercices et corrections au tableau
  \end{itemize}
\end{frame}

\begin{frame}{Objectifs}
  \begin{itemize}
    \item Revoir les mathématiques de base requises (algèbre linéaire)\\~\\
    \item Comprendre les notions essentielles de mécanique quantique\\~\\
    \item Programmer avec Qiskit\\~\\
    \item Aperçu des plateformes physiques pour implémenter un ordinateur quantique
  \end{itemize}
\end{frame}

\end{document}
