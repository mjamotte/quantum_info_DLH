\documentclass[aspectratio=169,12pt]{beamer}
\usetheme{Madrid}
\usecolortheme{dolphin}
\usefonttheme{professionalfonts}
\setbeamertemplate{navigation symbols}{}
\setbeamertemplate{footline}{}

\usepackage[T1]{fontenc}
\usepackage[utf8]{inputenc}
\usepackage{graphicx}
\usepackage{amsmath,amssymb}
\usepackage{hyperref}
\usepackage{siunitx}
\usepackage{physics}
\usepackage{braket}
\graphicspath{{../2-qubit/figures/}{../3-portes_logiques/figures/}{../4-qiskit_base/figures/}{../5-mesure/figures/}{../6-plusieurs_qubits/figures/}}

\title[]{Téléportation quantique}
\author{JAMOTTE Maxime, SCHOONEN Cédric}
\institute{Digital Learning Hub}
\date{}

\begin{document}

\begin{frame}
  \titlepage
\end{frame}

\begin{frame}{Rappel: paire de Bell, $C_X^{}$ et $C_Z^{}$}
  \begin{columns}[T,onlytextwidth]
    \column{0.6\textwidth}
    \begin{itemize}
      \item Créer la paire de Bell $\frac{1}{\sqrt 2}(\ket{00} + \ket{11})$.\\~\\
      \item Porte $C_X^{}$: applique $X$ sur un qubit si l'autre est dans l'état $\ket{1}$, sinon on ne fait rien.\\~\\
      \item Porte $C_Z^{}$: applique $Z$ sur un qubit si l'autre est dans l'état $\ket{1}$, sinon on ne fait rien.
    \end{itemize}
    \column{0.35\textwidth}
    \centering
    \includegraphics[width=0.6\linewidth]{figures/bell_pair_crop.png}
  \end{columns}
\end{frame}

\begin{frame}{Circuit pour téléporter}
  \begin{itemize}
    \item Alice possède deux qubit: $\ket \psi$ et un des qubits intriqués.
    \item Bob possède l'autre qubit intriqué.
    \item Les qubits intriqués forment la paire de Bell $\frac{1}{\sqrt 2}(\ket{00} + \ket{11})$.
  \end{itemize}
  \begin{figure}
    \includegraphics[width=\linewidth]{figures/teleportation.png}
  \end{figure}
\end{frame}

\begin{frame}{Circuit alternatif compatible avec un QPU}
    \begin{itemize}
      \item On veut implémenter ce circuit sur un QPU.\\~\\
      \item Les commandes telles que \texttt{if} en sont pas implémentables comme dans un ordinateur classique (pas du Python).\\~\\
      \item Il faut donc produire un circuit alternatif, où les mesures se font à la fin du circuit.
    \end{itemize}
\end{frame}

\begin{frame}{Circuit alternatif compatible avec un QPU}
    \centering Faire un pull (Github)
\end{frame}

\begin{frame}{Exercice: téléportation  et run sur QPU}
    \begin{itemize}
      \item Téléportez l'état $\ket \psi = \ket 1$ en utilisant la paire de Bell $\frac{1}{\sqrt 2}(\ket{00} + \ket{11})$ 
      et portes $C_X$ et $C_Z$.\\~\\
      \item Mesurer les trois qubits à la fin du circuit.\\~\\
      \item Testez votre circruit avec \texttt{Aer} (1000 shots).\\~\\
      \item Présentez les résulats dans un histogramme et interprétez les résulats.\\~\\
      \item Après vérification, envoyez votre circuit sur un QPU.
    \end{itemize}
\end{frame}  

\begin{frame}{Circuit qui tournera sur le QPU}
    \begin{figure}
    \includegraphics[width=\linewidth]{figures/teleportation_alternatif.png}
  \end{figure}
\end{frame}  
     

\end{document}
